\documentclass[12pt,a4paper]{article}

%% ── Packages ──────────────────────────────────────────────────────────────────
\usepackage[margin=2.5cm, headheight=15pt]{geometry}
\usepackage{amsmath, amssymb, amsthm}
\usepackage{mathtools}
\usepackage{booktabs}
\usepackage{array}
\usepackage{xcolor}
\usepackage{enumitem}
\usepackage{hyperref}
\usepackage{microtype}
\usepackage{graphicx}
\usepackage{float}
\usepackage[most]{tcolorbox}
\usepackage{parskip}
\usepackage{setspace}
\usepackage{fancyhdr}

%% ── Colours ───────────────────────────────────────────────────────────────────
\definecolor{laymanblue}{RGB}{215,232,252}
\definecolor{laymanframe}{RGB}{60,120,200}
\definecolor{mathgreen}{RGB}{228,248,228}
\definecolor{mathgreenframe}{RGB}{40,140,40}
\definecolor{warnbg}{RGB}{255,244,215}
\definecolor{warnframe}{RGB}{190,130,0}
\definecolor{examplebg}{RGB}{245,235,255}
\definecolor{exampleframe}{RGB}{130,60,180}
\definecolor{keyblue}{RGB}{30,80,160}
\definecolor{keygray}{RGB}{90,90,90}

%% ── tcolorbox environments ────────────────────────────────────────────────────
\tcbuselibrary{breakable, skins}

\newtcolorbox{laymanbox}[1][Intuition]{
  breakable, enhanced,
  colback=laymanblue, colframe=laymanframe,
  coltitle=white, fonttitle=\bfseries\small,
  title={$\bigstar$\enspace #1},
  arc=4pt, boxrule=1.2pt,
  left=6pt, right=6pt, top=4pt, bottom=4pt,
}

\newtcolorbox{mathbox}[1][Mathematical Detail]{
  breakable, enhanced,
  colback=mathgreen, colframe=mathgreenframe,
  coltitle=white, fonttitle=\bfseries\small,
  title={$\Sigma$\enspace #1},
  arc=4pt, boxrule=1.2pt,
  left=6pt, right=6pt, top=4pt, bottom=4pt,
}

\newtcolorbox{cautionbox}[1][Caveat / Key Assumption]{
  breakable, enhanced,
  colback=warnbg, colframe=warnframe,
  coltitle=white, fonttitle=\bfseries\small,
  title={$\triangle$\enspace #1},
  arc=4pt, boxrule=1.2pt,
  left=6pt, right=6pt, top=4pt, bottom=4pt,
}

\newtcolorbox{examplebox}[1][Worked Example]{
  breakable, enhanced,
  colback=examplebg, colframe=exampleframe,
  coltitle=white, fonttitle=\bfseries\small,
  title={$\blacktriangleright$\enspace #1},
  arc=4pt, boxrule=1.2pt,
  left=6pt, right=6pt, top=4pt, bottom=4pt,
}

%% ── Theorem styles ────────────────────────────────────────────────────────────
\theoremstyle{definition}
\newtheorem{defn}{Definition}[section]
\newtheorem{prop}{Proposition}[section]

%% ── Macros ────────────────────────────────────────────────────────────────────
\newcommand{\Ecorr}[2]{\mathbb{E}(#1,\,#2)}
\newcommand{\Sone}{S_1}
\newcommand{\abs}[1]{\left|#1\right|}
\newcommand{\sgn}{\operatorname{sign}}
\newcommand{\ind}[1]{\mathbf{1}[#1]}

%% ── Page style ────────────────────────────────────────────────────────────────
\pagestyle{fancy}
\fancyhf{}
\rhead{\small Bell--CHSH $S_1$ in Financial Markets}
\lhead{\small Step-by-Step Derivation}
\rfoot{\small\thepage}
\renewcommand{\headrulewidth}{0.4pt}

\hypersetup{
  colorlinks=true,
  linkcolor=keyblue, urlcolor=keyblue, citecolor=keyblue,
  pdftitle={Bell-CHSH S1: A Complete Step-by-Step Explanation},
}

\setstretch{1.15}

%% ═══════════════════════════════════════════════════════════════════════════════
\begin{document}
%% ═══════════════════════════════════════════════════════════════════════════════

\begin{titlepage}
\centering\vspace*{3cm}
{\LARGE\bfseries\color{keyblue}
  The Bell--CHSH $S_1$ Statistic in Financial Markets\\[0.8em]}
{\large A Complete, Step-by-Step Derivation\\
with Mathematical Justifications and Intuitive Explanations}\\[2.5cm]
{\large\color{keygray}
  Companion document to:\\[0.4em]
  \textit{Bell Inequality Violations in Agriculture Stock Networks}\\[0.5em]
  \texttt{github.com/mjpuma/StocksBellTest}}\\[3cm]
{\normalsize\color{keygray}
  Based on the methodology of Zarifian et al.\ (2025),\\
  \textit{Journal of Finance and Data Science}, 100164.}\\[2cm]
{\small\color{keygray} February 2026}
\end{titlepage}

\tableofcontents
\newpage

%% ── Section 1 ─────────────────────────────────────────────────────────────────
\section{Why Are We Doing This? The Big Picture}

\begin{laymanbox}[The Problem in Plain English]
Standard tools for measuring how two stocks move together---like Pearson
correlation---have a fundamental weakness: they cannot tell the difference
between two stocks that are correlated because of a \emph{shared hidden cause}
(both react to rising oil prices) versus two stocks whose co-movement is
structurally deeper, in a way that no single hidden cause can explain.

During financial crises, markets do not just become \emph{more} correlated;
they become correlated in a qualitatively different way. The Bell--CHSH test
is a tool borrowed from quantum physics that can detect this qualitative
difference. If the $S_1$ statistic exceeds~2, we have mathematical proof
that ordinary ``shared hidden-variable'' stories are insufficient.
\end{laymanbox}

The Bell--CHSH inequality was originally developed by John Bell (1964) and
Clauser, Horne, Shimony, and Holt (1969) to probe whether quantum mechanics
is fundamentally different from classical physics. The central insight is that
\emph{any} classical model with a shared hidden variable must satisfy a specific
algebraic bound. Violating that bound is a rigorous signal that the model is wrong.

We apply this logic to pairs of agricultural stocks. The question is not
``are these two stocks correlated?'' but rather ``is the pattern of co-movement
consistent with \emph{any} classical hidden-variable story?'' If $|S_1| > 2$,
the answer is no---and that violation is our crisis indicator.

\medskip
\noindent\textbf{Road map.} We walk through every component in sequence:

\begin{enumerate}[label=\textbf{Step \arabic*.}, leftmargin=*, itemsep=2pt]
  \item Raw data: stock returns
  \item Converting returns to binary outcomes (signs)
  \item Defining measurement regimes via threshold masks
  \item Computing four conditional expectation values
  \item Combining them into $S_1$
  \item The Bell bound: why $|S_1|>2$ is special
  \item Rolling windows: turning $S_1$ into a time series
  \item Aggregating across pairs: the violation percentage
\end{enumerate}

%% ── Section 2 ─────────────────────────────────────────────────────────────────
\section{Step 1 --- Raw Data: Stock Returns}

\subsection{What we start with}

We work with daily adjusted close prices $P_{i,t}$ for 53 agriculture-related
stocks (ticker index $i$, trading date $t$), covering 2000-01-01 to the present.

\subsection{Computing percentage returns}

\begin{defn}[Daily return]
  \begin{equation}\label{eq:return}
    r_{i,t} = \frac{P_{i,t} - P_{i,t-1}}{P_{i,t-1}}
  \end{equation}
\end{defn}

\begin{mathbox}[Why percentage returns, not log returns?]
Log returns $\ln(P_t / P_{t-1})$ are popular because they are additive over
time. For daily moves, $|r| \ll 1$ in normal conditions, and
$\ln(1+r) \approx r$, so both measures are nearly identical. Our pipeline uses
only the \emph{sign} and whether the \emph{magnitude} exceeds~5\%. Both
properties are identical for log and percentage returns at any realistic daily
frequency. Percentage returns are used for transparency and interpretability.
\end{mathbox}

\begin{laymanbox}[Returns in plain English]
A return of $r = 0.03$ means the stock rose 3\% that day.
A return of $r = -0.07$ means it fell 7\%. We will shortly reduce each return
to just two pieces of information: (a)~did it go up or down, and (b)~was the
move large (above~5\%) or small?
\end{laymanbox}

\subsection{Stock pairs}

We form all unordered pairs $(A, B)$ with $A \neq B$. For $n = 53$ tickers:
\[
  \binom{53}{2} = 1{,}378 \text{ pairs}
\]
All subsequent steps operate on a single pair $(A, B)$ over a rolling window.

%% ── Section 3 ─────────────────────────────────────────────────────────────────
\section{Step 2 --- Binary Outcomes: The Sign of Returns}

\subsection{Why binary outcomes?}

The Bell--CHSH inequality is stated for \textbf{binary} outcomes: each
measurement must yield either $+1$ or $-1$. In physics this corresponds to
spin measurements (spin up or spin down). In our financial application, the
natural binary is whether a stock rose or fell on a given day.

\begin{defn}[Binary outcome]
  \begin{equation}\label{eq:sign}
    a_i = \sgn(x_i) =
    \begin{cases}
      +1 & x_i > 0 \quad (\text{stock rose}) \\
      -1 & x_i < 0 \quad (\text{stock fell}) \\
       0 & x_i = 0 \quad (\text{excluded})
    \end{cases}
  \end{equation}
  and analogously $b_i = \sgn(y_i)$ for stock $B$, where $x_i = r_{A,i}$
  and $y_i = r_{B,i}$ within the rolling window.
\end{defn}

\begin{laymanbox}[Signs in plain English]
We simply ask: did the stock go up or down today?\\[3pt]
Up $\to +1$. \quad Down $\to -1$. \quad Unchanged $\to$ ignored.\\[6pt]
The \emph{product} $a_i b_i$ tells us whether both stocks moved the same
way ($+1 \cdot +1 = +1$ or $-1 \cdot -1 = +1$) or in opposite directions
($+1 \cdot -1 = -1$).
\end{laymanbox}

\subsection{Interpreting the product \texorpdfstring{$a_i b_i$}{a\_i b\_i}}

\begin{align*}
  a_i b_i = +1 &\implies \text{same direction (co-movement)}\\
  a_i b_i = -1 &\implies \text{opposite directions (divergence)}
\end{align*}

A simple average of $a_i b_i$ over many days gives a correlation-like measure
in $[-1, +1]$. Bell--CHSH goes further by conditioning on whether moves were
large or small---and this conditioning is what creates the statistical power
to detect non-classical behavior.

%% ── Section 4 ─────────────────────────────────────────────────────────────────
\section{Step 3 --- Measurement Settings: The Four Threshold Masks}

\subsection{Why we need ``settings''}

In the original Bell experiment, Alice chooses a measurement setting
$x \in \{0, 1\}$ and Bob chooses $y \in \{0, 1\}$, creating four setting
combinations $(x, y) \in \{(0,0),(0,1),(1,0),(1,1)\}$. The CHSH inequality
uses all four to form $S_1$.

For stocks, we define two ``settings'' via the \emph{magnitude} of returns:
\begin{itemize}
  \item Setting~$0$ (large / ``strong''): $|r| \geq \tau$
  \item Setting~$1$ (small / ``weak''): $|r| < \tau$
\end{itemize}
with fixed threshold $\tau = 0.05$ (5\%). This creates four regime combinations.

\begin{laymanbox}[The four regimes in plain English]
  Every day, each stock makes either a big move ($\geq 5\%$) or a small move
  ($< 5\%$). The pair of stocks together fall into one of four regimes:

  \medskip
  \begin{center}
  \renewcommand{\arraystretch}{1.3}
  \begin{tabular}{lllc}
    \toprule
    Stock $A$ & Stock $B$ & Regime & Notation \\
    \midrule
    Big ($\geq 5\%$) & Big ($\geq 5\%$) & both large & $ab$ \\
    Big ($\geq 5\%$) & Small ($< 5\%$)  & A large, B small & $ab'$ \\
    Small ($< 5\%$)  & Big ($\geq 5\%$) & A small, B large & $a'b$ \\
    Small ($< 5\%$)  & Small ($< 5\%$)  & both small & $a'b'$ \\
    \bottomrule
  \end{tabular}
  \end{center}

  \medskip
  Each trading day belongs to exactly one regime (they are mutually exclusive
  and exhaustive). The Bell test asks: does the direction agreement (up/down)
  differ systematically across these four regimes in a way that no hidden
  common factor can explain?
\end{laymanbox}

\subsection{Formal definition of the four masks}

\begin{defn}[Threshold masks]\label{def:masks}
  Fix $\tau = 0.05$. For each day $i$ in the window:
  \begin{align}
    m_{ab,i}   &= \ind{|x_i|\geq\tau}\cdot\ind{|y_i|\geq\tau}
                 \quad\text{(both large)}\label{eq:mab}\\
    m_{ab',i}  &= \ind{|x_i|\geq\tau}\cdot\ind{|y_i|<\tau}
                 \quad\text{(A large, B small)}\label{eq:mabp}\\
    m_{a'b,i}  &= \ind{|x_i|<\tau}\cdot\ind{|y_i|\geq\tau}
                 \quad\text{(A small, B large)}\label{eq:mapb}\\
    m_{a'b',i} &= \ind{|x_i|<\tau}\cdot\ind{|y_i|<\tau}
                 \quad\text{(both small)}\label{eq:mapbp}
  \end{align}
\end{defn}

\begin{mathbox}[Partition property]
  The four masks partition each day $i$ exactly:
  \[
    m_{ab,i} + m_{ab',i} + m_{a'b,i} + m_{a'b',i} = 1 \quad \forall i
  \]
  This holds because $\ind{|x|\geq\tau} + \ind{|x|<\tau} = 1$ for any
  $x \neq 0$, and taking the Cartesian product over $x$ and $y$ gives four
  disjoint events that together cover the entire sample space.
  (Days with $x_i = 0$ or $y_i = 0$ have $\sgn = 0$ and are excluded.)
\end{mathbox}

\subsection{Justification of \texorpdfstring{$\tau = 0.05$}{tau = 0.05}}

\begin{enumerate}[label=(\roman*), itemsep=4pt]
  \item \textbf{Economic meaning.} A 5\% daily move is approximately 2--3
  standard deviations of typical daily stock volatility ($\sigma_{\text{daily}}
  \approx 1.5\text{--}2.5\%$). This threshold separates routine trading from
  genuine event-driven price changes---the kind driven by earnings surprises,
  geopolitical shocks, or contagion during crises.

  \item \textbf{Non-degeneracy.} With a 20-day window, $\tau = 0.05$ typically
  places 1--5 observations in each regime cell, giving a non-trivial but
  computationally feasible split. If $\tau$ is too low, almost every day
  qualifies as ``large'' and the ``small'' regime degenerates (too few
  observations). If $\tau$ is too high, the ``large'' regime degenerates.

  \item \textbf{Literature consistency.} Zarifian et al.\ (2025) use the same
  threshold and confirm qualitative robustness across $\tau \in [0.01, 0.05]$.
\end{enumerate}

\begin{cautionbox}[The threshold is a modelling choice, not a universal law]
  The Bell \emph{bound} of~2 is derived mathematically and is independent of
  $\tau$. But the \emph{values} of $S_1$ depend on $\tau$. Results should
  always be reported with a sensitivity analysis varying $\tau$.
\end{cautionbox}

%% ── Section 5 ─────────────────────────────────────────────────────────────────
\section{Step 4 --- The Four Conditional Expectation Values}

\subsection{Rolling windows}

All computations use a rolling window of $W = 20$ trading days ending on date
$t$, containing observations $i = 1, \ldots, 20$. This window is the same
length as the rolling realized volatility computation, making comparisons direct.

\begin{laymanbox}[Why 20 days?]
  Twenty trading days is roughly one calendar month. It is long enough to
  populate each of the four regime cells with at least a few observations
  (needed for stable estimates) and short enough to respond quickly to
  emerging market stress. Longer windows would smooth out the crisis spikes;
  shorter windows would be too noisy.
\end{laymanbox}

\subsection{Computing each expectation}

\begin{defn}[Conditional expectation]\label{def:expect}
  For mask $m_{ab}$:
  \begin{equation}\label{eq:Eab}
    \Ecorr{a}{b} = \frac{\displaystyle\sum_{i=1}^{W} a_i\,b_i\,m_{ab,i}}
                        {\displaystyle\sum_{i=1}^{W} m_{ab,i}}
  \end{equation}
  and analogously $\Ecorr{a}{b'}$, $\Ecorr{a'}{b}$, $\Ecorr{a'}{b'}$ using
  masks $m_{ab'}$, $m_{a'b}$, $m_{a'b'}$ respectively.
  If the denominator is zero, the expectation is undefined and the pair
  is excluded from the violation count for date $t$.
\end{defn}

\begin{mathbox}[Unpacking Equation~\eqref{eq:Eab}]
  \textbf{Numerator:} $\sum_i a_i b_i m_{ab,i}$

  Sums the sign products over days when \emph{both} stocks made large moves.
  On days when at least one stock made a small move, $m_{ab,i} = 0$ so
  those days contribute nothing.

  \textbf{Denominator:} $\sum_i m_{ab,i}$

  Counts how many days in the window had both stocks making large moves.
  This is the effective sample size for this regime.

  \textbf{Result:} The ratio lies in $[-1, +1]$ and equals the average
  direction agreement \emph{among large-large days only}:
  \begin{align*}
    \Ecorr{a}{b} = +1 &\implies \text{whenever both moved a lot, they moved the same way}\\
    \Ecorr{a}{b} = -1 &\implies \text{whenever both moved a lot, they moved opposite ways}\\
    \Ecorr{a}{b} = 0  &\implies \text{no consistent pattern in large-large days}
  \end{align*}
\end{mathbox}

\begin{examplebox}[Numerical Example: 5 days (simplified)]
  \begin{center}
  \renewcommand{\arraystretch}{1.25}
  \begin{tabular}{cccccccc}
    \toprule
    Day & $x_i$ & $y_i$ & $a_i$ & $b_i$ & $a_ib_i$ &
    $|x|\!\geq\!0.05$ & $|y|\!\geq\!0.05$ \\
    \midrule
    1 & $+0.07$ & $+0.06$ & $+1$ & $+1$ & $+1$ & \checkmark & \checkmark \\
    2 & $-0.08$ & $-0.09$ & $-1$ & $-1$ & $+1$ & \checkmark & \checkmark \\
    3 & $+0.02$ & $+0.03$ & $+1$ & $+1$ & $+1$ & $\times$   & $\times$   \\
    4 & $+0.06$ & $-0.01$ & $+1$ & $-1$ & $-1$ & \checkmark & $\times$   \\
    5 & $-0.07$ & $+0.08$ & $-1$ & $+1$ & $-1$ & \checkmark & \checkmark \\
    \bottomrule
  \end{tabular}
  \end{center}

  \medskip
  \textbf{$\Ecorr{a}{b}$} (mask $m_{ab}$: both large): Days 1, 2, 5.
  \[
    \Ecorr{a}{b} = \frac{(+1)+(+1)+(-1)}{3} = \frac{1}{3} \approx +0.33
  \]

  \textbf{$\Ecorr{a}{b'}$} (mask $m_{ab'}$: A large, B small): Day 4 only.
  \[
    \Ecorr{a}{b'} = \frac{-1}{1} = -1
  \]

  \textbf{$\Ecorr{a'}{b}$} (mask $m_{a'b}$: A small, B large): No days.
  $\Rightarrow$ \emph{undefined}; pair excluded from violation count.

  \textbf{$\Ecorr{a'}{b'}$} (mask $m_{a'b'}$: both small): Day 3 only.
  \[
    \Ecorr{a'}{b'} = \frac{+1}{1} = +1
  \]
\end{examplebox}

%% ── Section 6 ─────────────────────────────────────────────────────────────────
\section{Step 5 --- Assembling \texorpdfstring{$S_1$}{S1}}

\subsection{The CHSH correlator}

\begin{defn}[Bell--CHSH $S_1$ correlator]\label{def:s1}
  \begin{equation}\label{eq:s1}
    \boxed{S_1 = \Ecorr{a}{b} + \Ecorr{a}{b'} + \Ecorr{a'}{b} - \Ecorr{a'}{b'}}
  \end{equation}
\end{defn}

\begin{laymanbox}[$S_1$ in plain English]
  We compute four ``direction-agreement scores,'' one for each regime:
  \begin{itemize}[itemsep=2pt]
    \item $\Ecorr{a}{b}$: agreement when \emph{both} stocks made big moves
    \item $\Ecorr{a}{b'}$: agreement when \emph{only A} made a big move
    \item $\Ecorr{a'}{b}$: agreement when \emph{only B} made a big move
    \item $\Ecorr{a'}{b'}$: agreement when \emph{neither} made a big move
  \end{itemize}
  We add the first three and \emph{subtract} the last. The reason for the
  minus sign is algebraic (see Section~\ref{sec:bound}): it is the combination
  that has a classical bound of exactly~2.
\end{laymanbox}

\subsection{Why this particular sign pattern \texorpdfstring{$(+,+,+,-)$}{(+,+,+,-)}?}

There are four CHSH variants, all of which obey $|S_k| \leq 2$ classically:
\begin{align}
  S_1 &= +\Ecorr{a}{b} + \Ecorr{a}{b'} + \Ecorr{a'}{b} - \Ecorr{a'}{b'} \\
  S_2 &= +\Ecorr{a}{b} + \Ecorr{a}{b'} - \Ecorr{a'}{b} + \Ecorr{a'}{b'} \\
  S_3 &= +\Ecorr{a}{b} - \Ecorr{a}{b'} + \Ecorr{a'}{b} + \Ecorr{a'}{b'} \\
  S_4 &= -\Ecorr{a}{b} + \Ecorr{a}{b'} + \Ecorr{a'}{b} + \Ecorr{a'}{b'}
\end{align}

\begin{mathbox}[Why $S_1$ is the preferred form for financial data]
  Rewriting $S_1$:
  \[
    S_1 = \underbrace{\Ecorr{a}{b} + \Ecorr{a}{b'} + \Ecorr{a'}{b}}_{\text{all involve at least one large move}} - \underbrace{\Ecorr{a'}{b'}}_{\text{both small}}
  \]
  The three positive terms capture correlations in \emph{at least one} extreme
  regime. The negative term is the background correlation in quiet periods.
  $S_1$ therefore measures \emph{excess} co-movement during volatile episodes
  relative to calm conditions---which is exactly what a crisis indicator should
  capture. The other three variants ($S_2, S_3, S_4$) mix large-move and
  small-move terms in both the positive and negative sums, losing this
  interpretation. This is the argument of Zarifian et al.\ (2025) for
  using $S_1$ exclusively.
\end{mathbox}

\subsection{Numerical range}

Each expectation term lies in $[-1, +1]$, so:
\[
  S_1 \in [-4,\, +4]
\]
because $|\!\pm 1 \pm 1 \pm 1 \mp (\mp 1)| \leq 4$.

%% ── Section 7 ─────────────────────────────────────────────────────────────────
\section{Step 6 --- The Bell Bound: Why \texorpdfstring{$|S_1| > 2$}{|S1| > 2} Is Special}
\label{sec:bound}

This is the mathematical heart of the method. We prove from first principles
that any classical shared-factor model must satisfy $|S_1| \leq 2$.

\subsection{What is a local hidden-variable model?}

\begin{defn}[Local Hidden-Variable (LHV) Model]\label{def:lhv}
  An LHV model for a stock pair assumes an unobserved random variable
  $\lambda$ (the ``hidden variable'' or ``common factor'') such that:
  \begin{enumerate}[label=(\roman*), itemsep=2pt]
    \item Stock $A$'s outcome depends only on $\lambda$ and $A$'s own regime
    indicator $x \in \{0,1\}$, not on stock $B$.
    \item Stock $B$'s outcome depends only on $\lambda$ and $B$'s own regime
    indicator $y \in \{0,1\}$, not on stock $A$.
    \item $\lambda$ can be of any form and any dimension.
  \end{enumerate}
  Formally, outcomes are generated by response functions $a(x,\lambda)\in\{-1,+1\}$
  and $b(y,\lambda)\in\{-1,+1\}$, and the joint probability satisfies:
  \begin{equation}\label{eq:lhv}
    P(a,b \mid x,y) = \int P(a \mid x,\lambda)\,P(b \mid y,\lambda)\,p(\lambda)\,d\lambda
  \end{equation}
\end{defn}

\begin{laymanbox}[LHV in plain English]
  An LHV model says: ``There is some shared cause $\lambda$---perhaps market
  sentiment, the Fed funds rate expectation, the dollar index, or anything
  else---that influences both stocks. Once you know $\lambda$, the two stocks
  behave \emph{independently} of each other. All correlation between them is
  entirely due to this shared cause.''

  This is the implicit assumption behind Pearson correlation: if two stocks
  are correlated, they share a common driver. Bell's theorem says:
  if $|S_1| > 2$, no such story works---no matter how many common factors
  you posit, no matter how complex.
\end{laymanbox}

\subsection{Proof that LHV models satisfy \texorpdfstring{$|S_1| \leq 2$}{|S1| <= 2}}

\begin{prop}[Bell--CHSH inequality]\label{prop:bell}
  Under any LHV model as in Definition~\ref{def:lhv},
  \[
    |S_1| \leq 2
  \]
\end{prop}

\begin{proof}
  For a fixed realization of $\lambda$, denote the four deterministic outcomes
  (one for each setting combination):
  \[
    a_0 = a(0,\lambda),\quad
    a_1 = a(1,\lambda),\quad
    b_0 = b(0,\lambda),\quad
    b_1 = b(1,\lambda),
    \quad a_k, b_k \in \{-1, +1\}
  \]
  The realized CHSH expression for this $\lambda$ is:
  \begin{equation}\label{eq:realized}
    s(\lambda) = a_0 b_0 + a_0 b_1 + a_1 b_0 - a_1 b_1
               = a_0(b_0 + b_1) + a_1(b_0 - b_1)
  \end{equation}
  Since $b_0, b_1 \in \{-1, +1\}$, exactly one of two cases holds:

  \medskip
  \textbf{Case 1: $b_0 = b_1$.}
  Then $b_0 + b_1 = \pm 2$ and $b_0 - b_1 = 0$, giving
  $s(\lambda) = a_0(\pm 2) + 0 = \pm 2$.

  \medskip
  \textbf{Case 2: $b_0 = -b_1$.}
  Then $b_0 + b_1 = 0$ and $b_0 - b_1 = \pm 2$, giving
  $s(\lambda) = 0 + a_1(\pm 2) = \pm 2$.

  \medskip
  In both cases $|s(\lambda)| = 2$ for every realization of $\lambda$.
  The observed $S_1$ is an average over $\lambda$:
  \[
    S_1 = \int s(\lambda)\, p(\lambda)\, d\lambda
  \]
  By the triangle inequality:
  \[
    |S_1| \leq \int |s(\lambda)|\, p(\lambda)\, d\lambda = 2\int p(\lambda)\,d\lambda = 2
  \qedhere\]
\end{proof}

\begin{mathbox}[The key algebraic insight]
  The proof shows that for \emph{any single} realization of the hidden variable,
  the expression $a_0 b_0 + a_0 b_1 + a_1 b_0 - a_1 b_1$ always equals exactly
  $\pm 2$---never more, never less. This is pure algebra, requiring only
  $b_0, b_1 \in \{-1, +1\}$. Averaging over $\lambda$ cannot push the result
  beyond $\pm 2$.

  Therefore, \emph{observing} $|S_1| > 2$ in data is a mathematical proof that
  the data cannot have come from any LHV model, regardless of how complex or
  numerous the hidden variables are.
\end{mathbox}

\subsection{The three classical and quantum bounds}

\begin{center}
\renewcommand{\arraystretch}{1.4}
\begin{tabular}{llc}
  \toprule
  Model class & Condition & Bound on $|S_1|$ \\
  \midrule
  Local hidden-variable (classical) & Any common-cause model & $\leq 2$ \\
  Quantum mechanics & Entangled quantum systems & $\leq 2\sqrt{2} \approx 2.83$ \\
  No-signaling (most general) & Any non-signaling theory & $\leq 4$ \\
  \bottomrule
\end{tabular}
\end{center}

\begin{laymanbox}[What the three bounds mean in practice]
  \begin{itemize}[itemsep=4pt]
    \item $|S_1| \leq 2$: The data are compatible with (though do not prove)
    a classical hidden-variable explanation. Normal market periods.

    \item $2 < |S_1| \leq 2\sqrt{2}$: No classical hidden-variable model fits.
    The co-movement is structurally ``more than classical.''
    In finance, this is our crisis signal: the pair exhibits synchronized
    extreme behavior that cannot be attributed to any shared common factor.

    \item $2\sqrt{2} < |S_1| \leq 4$: Even quantum mechanics (with the
    standard CHSH setup) cannot explain this. In practice, this range
    typically reflects statistical noise from sparse regime cells (too few
    observations in a 20-day window). It should be interpreted cautiously.
  \end{itemize}
\end{laymanbox}

\begin{cautionbox}[We are NOT claiming stocks are quantum systems]
  When we say ``Bell violation,'' we do \emph{not} claim quantum entanglement
  in financial markets. The Bell inequality is a constraint on
  \emph{classical probability theory}, not only on quantum mechanics. The
  claim is:

  \begin{quote}
    \itshape The conditional joint return distribution of this stock pair,
    across these four magnitude regimes, cannot be reproduced by any model
    in which the two stocks respond independently to a shared common factor.
  \end{quote}

  This is a statement about probability structure. The Bell framework is used
  because it provides the mathematical tool to detect this structural property
  rigorously, with a threshold (the bound of~2) that follows from first
  principles rather than being a tuning parameter.
\end{cautionbox}

%% ── Section 8 ─────────────────────────────────────────────────────────────────
\section{Step 7 --- Rolling Windows: \texorpdfstring{$S_1$}{S1} as a Time Series}

\subsection{Computing \texorpdfstring{$S_1(t)$}{S1(t)} for each date \texorpdfstring{$t$}{t}}

For each pair $(A,B)$ and each date $t$, we apply the 20-day rolling window
ending on $t$ and compute $S_1^{(A,B)}(t)$ as in Definition~\ref{def:s1}.
This yields:
\[
  S_1^{(A,B)}(t) \in [-4, +4] \quad \text{for each pair }(A,B)\text{ and date }t
\]

\begin{cautionbox}[Statistical precision in a 20-day window]
  With $W = 20$ observations split across 4 regime cells, each cell contains
  on average only 5 observations. Individual $S_1$ values will be noisy. This
  is acceptable because:

  \begin{enumerate}[label=(\alph*), itemsep=2pt]
    \item We aggregate over 1,378 pairs---noise at the pair level averages out
    in the violation percentage.
    \item Short windows are deliberately chosen to be \emph{responsive}: a
    60-day window would be more statistically stable but would detect crises
    weeks later, reducing practical utility as an early-warning indicator.
    \item Noise is symmetric around the true $S_1$, so pairs should not
    \emph{systematically} exceed 2 by chance. This can be verified by comparing
    observed violation rates to a permutation-test null.
  \end{enumerate}
\end{cautionbox}

\subsection{Handling undefined expectations}

If any regime cell contains zero observations (denominator~$= 0$), $S_1$ is
undefined for that pair on that date. Such pairs are:
\begin{itemize}[itemsep=2pt]
  \item Excluded from the numerator of violation percentage (not counted as
  either violating or non-violating)
  \item Excluded from the denominator (do not inflate TotalPairs)
\end{itemize}
This ensures the violation percentage is always computed over valid pairs only.

%% ── Section 9 ─────────────────────────────────────────────────────────────────
\section{Step 8 --- The Violation Percentage}

\subsection{Definition}

\begin{defn}[Daily violation percentage]\label{def:viol}
  Let $N_t^{\text{viol}}$ be the number of pairs with $|S_1^{(A,B)}(t)| > 2$
  and valid data. Let $\text{TotalPairs}_t$ be the number of pairs with valid
  $S_1$. Then:
  \begin{equation}\label{eq:viol}
    \text{ViolationPct}(t)
    = 100 \times \frac{N_t^{\text{viol}}}{\text{TotalPairs}_t}
  \end{equation}
\end{defn}

\begin{laymanbox}[Violation percentage in plain English]
  On any given day we look at all $\sim$1,378 stock pairs. For each pair,
  we ask: is $|S_1| > 2$? We count the fraction that say ``yes.''

  During calm periods: few pairs violate; the percentage is low.

  During crises (2008, COVID-19, Ukraine War): many pairs simultaneously
  exhibit non-classical co-movement; the percentage spikes sharply.

  Plotting this one number against time gives a crisis indicator that can
  be compared directly with VIX, GARCH volatility, and CDS spreads.
\end{laymanbox}

\subsection{From scalar to network}

The violation percentage is a single scalar per day. Our paper's key extension
is to treat the pairs $(A,B)$ with $|S_1| > 2$ as \emph{edges} in a daily
network:
\[
  G_t = (\mathcal{V},\, \mathcal{E}_t), \qquad
  \mathcal{E}_t = \{(A,B) : \abs{S_1^{(A,B)}(t)} > 2\}
\]
The topology of $G_t$---its density, clustering, giant component size,
community structure, hub structure---characterizes the \emph{architecture}
of the crisis, not just its intensity.

\begin{laymanbox}[The network picture in plain English]
  Think of the 53 stocks as cities. An edge (road) exists between two cities
  when their Bell statistic exceeds~2---meaning their co-movement is too
  tightly synchronized to be explained by any common factor.

  On a quiet day: few roads; cities are mostly isolated.

  During a crisis: roads appear everywhere simultaneously. The resulting
  ``crisis network'' can be analysed:
  \begin{itemize}[itemsep=2pt]
    \item \textbf{High density}: almost all pairs are synchronized
    \item \textbf{High clustering}: violations cluster within sectors
    \item \textbf{Few communities}: the market consolidates into fewer, larger
    synchronized blocs
    \item \textbf{Hub emergence}: certain stocks become central connectors
  \end{itemize}
  These topological features carry information that the scalar violation
  percentage alone cannot reveal.
\end{laymanbox}

%% ── Section 10 ────────────────────────────────────────────────────────────────
\section{Complete Summary}

\begin{tcolorbox}[
  breakable, enhanced,
  colback=white, colframe=keyblue,
  coltitle=white, fonttitle=\bfseries,
  title={The Complete $S_1$ Computation at a Glance},
  arc=5pt, boxrule=1.5pt,
  left=10pt, right=10pt, top=8pt, bottom=8pt,
]

\textbf{Input:} Adjusted close prices for stocks $A$ and $B$,
rolling 20-day window ending on date $t$.

\bigskip
\textbf{Step 1: Daily returns.}
\[
  r_{i} = \frac{P_{i} - P_{i-1}}{P_{i-1}}, \quad i = 1,\ldots,20
\]

\textbf{Step 2: Binary signs.}
\[
  a_i = \sgn(r_{A,i}), \qquad b_i = \sgn(r_{B,i})
\]

\textbf{Step 3: Threshold masks} ($\tau = 0.05$):
\begin{align*}
  m_{ab,i}   &= \ind{|r_{A,i}|\geq\tau}\cdot\ind{|r_{B,i}|\geq\tau} \\
  m_{ab',i}  &= \ind{|r_{A,i}|\geq\tau}\cdot\ind{|r_{B,i}|<\tau} \\
  m_{a'b,i}  &= \ind{|r_{A,i}|<\tau}\cdot\ind{|r_{B,i}|\geq\tau} \\
  m_{a'b',i} &= \ind{|r_{A,i}|<\tau}\cdot\ind{|r_{B,i}|<\tau}
\end{align*}

\textbf{Step 4: Conditional expectations.}
\[
  \Ecorr{a}{b} = \frac{\sum_i a_i b_i m_{ab,i}}{\sum_i m_{ab,i}}, \quad
  \Ecorr{a}{b'} = \frac{\sum_i a_i b_i m_{ab',i}}{\sum_i m_{ab',i}},
  \quad \text{etc.}
\]

\textbf{Step 5: CHSH correlator.}
\[
  \boxed{S_1 = \Ecorr{a}{b} + \Ecorr{a}{b'} + \Ecorr{a'}{b} - \Ecorr{a'}{b'}}
\]

\textbf{Step 6: Classify.}
\[
  \text{Violation} = \ind{|S_1| > 2}, \quad
  \text{Classical bound: }|S_1|\leq 2 \text{ (proven)}
\]

\textbf{Step 7: Aggregate over all pairs.}
\[
  \text{ViolationPct}(t) = 100 \times
  \frac{\#\{\text{pairs with }|S_1|>2\}}{\#\{\text{pairs with valid }S_1\}}
\]

\end{tcolorbox}

%% ── Section 11 ────────────────────────────────────────────────────────────────
\section{Addressing Common Objections}

\subsection{``This is just Pearson correlation in disguise''}

No. Pearson correlation is an \emph{unconditional} average of co-movement
across all days. $S_1$ computes \emph{conditional} averages within each of four
magnitude regimes and combines them with a specific sign pattern. Even if two
stocks have high Pearson correlation, their $S_1$ may remain within $[-2, +2]$.
Conversely, two stocks with low Pearson correlation could still violate if the
regime-conditional structure is non-classical. The algebraic bound of~2 has
no analogue for Pearson correlation; it is specific to the CHSH combination.

\subsection{``Shared market shocks explain all the correlation''}

This is the LHV claim, and Bell's theorem refutes it when $|S_1| > 2$. Any
model in which both stocks respond (however complexly) to a shared factor
$\lambda$---a Fed announcement, a crop report, a geopolitical event---is by
definition an LHV model and therefore satisfies $|S_1| \leq 2$.
Observation of $|S_1| > 2$ is mathematical proof that no such model fits.

\subsection{``The 5\% threshold is arbitrary''}

The value $\tau = 0.05$ is a design choice. The Bell bound of~2 is not---it
is derived algebraically and holds for any threshold. Sensitivity analyses
confirm robustness across $\tau \in [0.01, 0.05]$.

\subsection{``Violations could arise from statistical noise in small windows''}

True: with $\sim$5 observations per regime cell, individual pair $S_1$ values
are noisy. But the aggregate violation percentage over 1,378 pairs is far more
stable. And crucially, noise should not produce \emph{systematic} bias toward
$|S_1| > 2$; this can be tested with a permutation null (randomly shuffling
return time series within pairs destroys true temporal structure while
preserving marginal distributions). The persistence of elevated violation
percentages specifically during known crisis periods, and not during calm
periods, is the key empirical validation.

%% ── References ────────────────────────────────────────────────────────────────
\section*{References}
\addcontentsline{toc}{section}{References}

\begin{enumerate}[label={[\arabic*]}, leftmargin=*, itemsep=5pt]
  \item Bell, J.\,S.\ (1964). On the Einstein Podolsky Rosen paradox.
        \textit{Physics Physique Fizika}, 1(3), 195--200.

  \item Clauser, J.\,F., Horne, M.\,A., Shimony, A., \& Holt, R.\,A.\ (1969).
        Proposed experiment to test local hidden-variable theories.
        \textit{Physical Review Letters}, 23(15), 880--884.

  \item Cirel'son, B.\,S.\ (1980). Quantum generalizations of Bell's inequality.
        \textit{Letters in Mathematical Physics}, 4(2), 93--100.

  \item Gallus, C., Blasiak, P., Pothos, E.\,M., Yearsley, J.\,M., \&
        Borsuk, E.\ (2023). Bell correlations outside physics.
        \textit{Scientific Reports}, 13, 4952.

  \item Zarifian, A., Gallus, C., Blasiak, P., Pothos, E., \&
        Overbeck, L.\ (2025). Using Bell violations as an indicator for
        financial crisis. \textit{Journal of Finance and Data Science},
        11, 100164.

  \item Busemeyer, J.\,R.\ \& Bruza, P.\,D.\ (2012).
        \textit{Quantum Models of Cognition and Decision}.
        Cambridge University Press.
\end{enumerate}

\end{document}
